%%%%%%%%%%%%%%%%%%%%%%%%%%%%%%%%%%%%%%%%%%%%%%%%%%%%%%%%%%%%%%%%%%%%%%%%%%%

\chapter{Number Theory}


%%%%%%%%%%%%%%%%%%%%%%%%%%%%%%%%%%%%%%%%%%%%%%%%%%%%%%%%%%%%%%%%%%%%%%%%%%%

\section{Integers}

An \emph{integer} is a whole number. It can be positive, negative, or
zero. Examples of integers include $-13$, $-1$, $0$, and $42$. When we
talk about all the whole numbers, we can use the name \emph{integer}
to refer to them, or we can list all the whole numbers like so:
\[
\dots, -4, -3, -2, -1, 0, 1, 2, 3, 4, \dots
\]
But for convenience, we usually write $\ZZ$ to denote the set of all
integers:
\[
\ZZ
=
\{\dots, -3, -2, -1, 0, 1, 2, 3, \dots\}.
\]
In Sage, the set of all integers is represented as \verb!ZZ! and a
single integer is represented using \verb!Integer!:
%
\begin{lstlisting}
sage: ZZ
Integer Ring
sage: type(ZZ)
<type 'sage.rings.integer_ring.IntegerRing_class'>
sage: 1 in ZZ
True
sage: -1 in ZZ
True
sage: 3.1415 in ZZ
False
sage: type(3)
<type 'sage.rings.integer.Integer'>
sage: Integer
<type 'sage.rings.integer.Integer'>
\end{lstlisting}
%
Notice that in the above code listing, we used \verb!in! to test
whether or not a number is an integer. The command
%
\begin{lstlisting}
sage: 1 in ZZ
True
\end{lstlisting}
%
gives the result \verb!True! because $1$ is an integer. On the other
hand, the command
%
\begin{lstlisting}
sage: 3.1415 in ZZ
False
\end{lstlisting}
%
gives \verb!False! because $3.1415$ is not an integer.
