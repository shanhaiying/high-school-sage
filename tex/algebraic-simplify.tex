%%%%%%%%%%%%%%%%%%%%%%%%%%%%%%%%%%%%%%%%%%%%%%%%%%%%%%%%%%%%%%%%%%%%%%%%%%%

\chapter{Algebraic Simplification}

\begin{enumerate}
\item Simplifying algebraic expressions using the distributive laws:
  \begin{enumerate}
  \item binomial expansion: $(a + b)(c + d) = ac + ad + bc + bd$

  \item difference of two squares: $(a + b)(a - b) = a^2 - b^2$

  \item perfect squares: $(a + b)^2 = a^2 + 2ab + b^2$ and
    $(a - b)^2 = a^2 - 2ab + b^2$

  \item perfect cubes

  \item sum and difference of two cubes
  \end{enumerate}

\item Simplifying rational expressions:
  \begin{enumerate}
  \item addition of rational expressions

  \item subtraction of rational expressions

  \item multiplication of rational expressions

  \item division of rational expressions
  \end{enumerate}
\end{enumerate}


%%%%%%%%%%%%%%%%%%%%%%%%%%%%%%%%%%%%%%%%%%%%%%%%%%%%%%%%%%%%%%%%%%%%%%%%%%%

\section{Collect like terms}
\index{like terms}

In simplifying an algebraic expression, a basic technique is
identifying like terms\index{like terms}, simplify them and in the
process we also simplify the whole expression. Two like terms can be
simplified to one term because only like terms can be added or
subtracted. The expression
%
\begin{equation}
\label{eq:algebraic_simplify:has_like_terms}
5x + 12x + 17y
\end{equation}
%
can be simplified because $5x$ and $12x$ are like
terms. Contrast~(\ref{eq:algebraic_simplify:has_like_terms}) with
\[
3x + 7y
\]
which has no like terms and cannot be simplified any further.

\begin{lstlisting}
sage: x, y = var("x, y")
sage: 5*x + 12*x + 17*y
17*x + 17*y
sage: simplify(5*x + 12*x + 17*y)
17*x + 17*y
sage: 3*x + 7*y
3*x + 7*y
sage: simplify(3*x + 7*y)
3*x + 7*y
sage: a, b, m, n = var("a, b, m, n")
sage: a + b^2 - 4*a*b + 3*a - b^2 - 2*a*b
-6*a*b + 4*a
sage: 3*(x^2)*(y^3) + 4*x*(y^2) - 9*(y^3)*(x^2) - 7*(y^2)*x
-6*x^2*y^3 - 3*x*y^2
\end{lstlisting}


%%%%%%%%%%%%%%%%%%%%%%%%%%%%%%%%%%%%%%%%%%%%%%%%%%%%%%%%%%%%%%%%%%%%%%%%%%%

\section{Distributive laws}
\index{distributive laws}

An expression such as
%
\begin{equation}
\label{eq:algebraic_simplify:simplify_with_left_distributive_law}
7 \times (9 + 1)
\end{equation}
%
can be calculated in various ways. We can first simplify the
expression within the parentheses to get $10$, then multiply this
result with the number outside the parentheses to get $70$:
\[
7 \times (9 + 1)
=
7 \times 10
=
70.
\]
An alternative method for
simplifying~(\ref{eq:algebraic_simplify:simplify_with_left_distributive_law})
is via the left distributive law\index{distributive law!left}, also
called ``expanding the brackets.'' Take the term to the left
of~(\ref{eq:algebraic_simplify:simplify_with_left_distributive_law})
and multiply that with each term inside the parentheses. Add up the
result to get a simplified expression:
\[
7 \times (9 + 1)
=
7 \times 9 + 7 \times 1
=
63 + 7
=
70.
\]
%
\begin{lstlisting}
sage: 7 * (9 + 1)
70
sage: 7*9 + 7*1
70
\end{lstlisting}
%
The left distributive law\index{distributive law!left} is
%
\begin{equation}
\label{eq:algebraic_simplify:left_distributive_law}
\begin{aligned}
a(b + c) &= ab + ac \\
a(b + c + d) &= ab + ac + ad.
\end{aligned}
\end{equation}

\begin{lstlisting}
sage: a, b, c = var("a, b, c")
sage: expand(a * (b + c))
a*b + a*c
sage: bool(a * (b + c) == a*b + a*c)
True
\end{lstlisting}

The right distributive law\index{distributive law!right} works
similarly to its left counterpart
in~(\ref{eq:algebraic_simplify:left_distributive_law}):
%
\begin{equation}
\label{eq:algebraic_simplify:right_distributive_law}
\begin{aligned}
(b + c)a &= ab + ac \\
(b + c + d)a &= ab + ac + ad.
\end{aligned}
\end{equation}
%
We take the term to the right of the parentheses, multiply it with
each term inside the parentheses, finally summing to get a simplified
expression. For example,
\[
(5 + 3) \times 9
=
5 \times 9 + 3 \times 9
=
45 + 27
=
72.
\]

\begin{lstlisting}
sage: (5 + 3) * 9
72
sage: (5*9) + (3*9)
72
sage: a, b, c = var("a, b, c")
sage: expand((b + c) * a)
a*b + a*c
\end{lstlisting}

The rules~(\ref{eq:algebraic_simplify:left_distributive_law})
and~(\ref{eq:algebraic_simplify:right_distributive_law}) are
collectively referred to as the distributive laws\index{distributive laws}.
These rules can be repeatedly applied to simplify expressions such as
\[
7(4x - 3) - 3(3 - 5x)
\]
and
\[
7b(b^2 + 2b - 3) + 3b(b^2 - 4b + 1).
\]

\begin{lstlisting}
sage: x, b = var("x, b")
sage: # first simplify 7(4x - 3) - 3(3 - 5x)
sage: f = expand(7*(4*x - 3)); f
28*x - 21
sage: g = expand(3*(3 - 5*x)); g
-15*x + 9
sage: simplify(f - g)
43*x - 30
sage: # now simplify 7b(b^2 + 2b - 3) + 3b(b^2 - 4b + 1)
sage: f = expand(7*b*(b^2 + 2*b - 3)); f
7*b^3 + 14*b^2 - 21*b
sage: g = expand(3*b*(b^2 - 4*b + 1)); g
3*b^3 - 12*b^2 + 3*b
sage: simplify(f + g)
10*b^3 + 2*b^2 - 18*b
\end{lstlisting}
