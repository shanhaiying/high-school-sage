%%%%%%%%%%%%%%%%%%%%%%%%%%%%%%%%%%%%%%%%%%%%%%%%%%%%%%%%%%%%%%%%%%%%%%%%%%%

\chapter{Algebraic Simplification}

\begin{enumerate}
\item Simplifying algebraic expressions using the distributive laws:
  \begin{enumerate}
  \item binomial expansion: $(a + b)(c + d) = ac + ad + bc + bd$

  \item difference of two squares: $(a + b)(a - b) = a^2 - b^2$

  \item perfect squares: $(a + b)^2 = a^2 + 2ab + b^2$ and
    $(a - b)^2 = a^2 - 2ab + b^2$

  \item perfect cubes

  \item sum and difference of two cubes
  \end{enumerate}

\item Simplifying rational expressions:
  \begin{enumerate}
  \item addition of rational expressions

  \item subtraction of rational expressions

  \item multiplication of rational expressions

  \item division of rational expressions
  \end{enumerate}
\end{enumerate}


%%%%%%%%%%%%%%%%%%%%%%%%%%%%%%%%%%%%%%%%%%%%%%%%%%%%%%%%%%%%%%%%%%%%%%%%%%%

\section{Collect like terms}

In simplifying an algebraic expression, a basic technique is
identifying like terms, simplify them and in the process we also
simplify the whole expression. Two like terms can be simplified to one
term because only like terms can be added or subtracted. The
expression
%
\begin{equation}
\label{eq:algebraic_simplify:has_like_terms}
5x + 12x + 17y
\end{equation}
%
can be simplified because $5x$ and $12x$ are like
terms. Contrast~(\ref{eq:algebraic_simplify:has_like_terms}) with
\[
3x + 7y
\]
which has no like terms and cannot be simplified any further.

\begin{lstlisting}
sage: x, y = var("x, y")
sage: 5*x + 12*x + 17*y
17*x + 17*y
sage: simplify(5*x + 12*x + 17*y)
17*x + 17*y
sage: 3*x + 7*y
3*x + 7*y
sage: simplify(3*x + 7*y)
3*x + 7*y
sage: a, b, m, n = var("a, b, m, n")
sage: a + b^2 - 4*a*b + 3*a - b^2 - 2*a*b
-6*a*b + 4*a
sage: 3*(x^2)*(y^3) + 4*x*(y^2) - 9*(y^3)*(x^2) - 7*(y^2)*x
-6*x^2*y^3 - 3*x*y^2
\end{lstlisting}
